\documentclass[10pt,a4paper]{article}
\usepackage[utf8]{inputenc}
\usepackage[T1]{fontenc}

\usepackage[
backend=bibtex,
sorting=ynt
]{biblatex}
\addbibresource{refs.bib}

\defbibheading{secbib}[\bibname]{%
	\section*{#1}%
	\markboth{#1}{#1}}

\usepackage{graphicx}
\usepackage{subfig}
\usepackage{todonotes}
\usepackage{amsfonts}
\usepackage{amsmath}
\usepackage{amssymb}
\usepackage{amsthm}
\usepackage{tabularx}
\usepackage{array}
\setlength\extrarowheight{2pt} % or whatever amount is appropriate
\usepackage{hyperref}
\usepackage[capitalise, noabbrev]{cleveref}
\usepackage{float}
\usepackage{mathbbol}
\usepackage{tikz}
\usepackage{tikz-cd}
\usepackage{enumitem}
\usepackage{xcolor}
\usepackage{tcolorbox}
\tcbuselibrary{breakable}

\usepackage{newpxtext,newpxmath} % Should be the last package import

\theoremstyle{definition}
\newtheorem{thm}{Theorem}[section]
\newtheorem{defn}[thm]{Definition}
\newtheorem{lem}{Lemma}[thm]
\newtheorem{cor}{Corollary}[thm]
\newtheorem{prop}{Proposition}[thm]
\newtheorem{rem}{Remark}[thm]
\newtheorem{ill}{Illustration}[thm]
\newtheorem{ex}{Example}[thm]

\newcommand{\R}{\mathbb{R}}
\newcommand{\N}{\mathbb{N}}
\newcommand{\Dgm}{\operatorname{Dgm}}


\setlength\parindent{0pt}
\linespread{1.5}
\title{Dowker dissimilarities}

\begin{document}
\maketitle

\section{Introduction}

In this note, we summarize our progress attempting to generalize the results in \autocite{blaser2019sparse} replacing $[0,\infty)$ by a poset $P$.

\begin{defn}[Poset stuff]
	For the following definitions, let $P$ and $Q$ be posets, and let $[1]$ denote the poset $0\leq 1$.
	\begin{itemize}
		\item The \textit{product poset} $P\times Q$ with $(p,q)\leq(p',q')\iff p\leq p'\text{ and }q\leq q'$.
		
		\item The \textit{extended poset} $P^*:=\hom(P,[1])$ with $\infty \equiv 1$.
		
		\item The \textit{opposite poset} $P^{op}$ with the same underlying set as $P$ but with the reverse order. That is, $p\leq p'$ in $P^{op}$ if and only if $p'\leq p$ in $P$.
	\end{itemize}
\end{defn} 

\begin{defn}[Dowker stuff]
	Some fundamental definitions related to Dowker dissimilarities.
	\begin{itemize}
		\item A \textit{Dowker dissimilarity} $\Lambda$ consists of two sets $L$ and $W$, and a function $\Lambda\colon L\times W\to P^*$ where $P$ is some poset.
		
		\item Given $l\in L$ and $p\in P$, define the \textit{ball of radius $p$ centred at $l$} to be the set $B_\Lambda(l,p):=\{w\in W\mid\Lambda(l,w)(p)=1\}$. 
		
		\item Given some $p$ in $P$, define $\Lambda_p:=\{(l,w)\in L\times W\mid w\in B_\Lambda(l,p)\}$.
		
		\item The \textit{Dowker nerve} $N\Lambda$ of $\Lambda$ is the collection of simplicial complexes indexed over $P$ defined in degree $p$ as 
		$$
		(N\Lambda)_{p} = \left\{\sigma\subseteq L\mid\sigma\text{ is finite and } \exists w\in W \text{ with } w\in B_\Lambda(l,p)\text{ for all }l\in\sigma \right\}.
		$$
		
		In other words, the Dowker nerve $(N\Lambda)_p$ is the nerve of the covering $\{B_\Lambda(l,p)\}_{l\in L}$.
	\end{itemize}
\end{defn} 

Note that if $p\leq p'$ for $p,p'\in P$, then we have an inclusion $N\Lambda_{p\leq p'}\colon N\Lambda_p\hookrightarrow N\Lambda_{p'}$.

\begin{ex}[Čech filtration]
	Consider the poset $P=[0,\infty)$ and let $L\subset W=\R^d$. If we set $\Lambda\colon L\times W\to P^*$ to be the dissimilarity defined for $t\in P$ by 
	$$
	\Lambda(l,w)(t) := \begin{cases}
		1\quad\text{if }t\geq d(l,w) \text{ and}\\
		0\quad\text{otherwise}
	\end{cases}
	$$
	then we have $\sigma\in N\Lambda_t$ if and only if there exists some $w\in\R^d$ with $d(l,w)\leq t$ for all $l\in\sigma$. The existence of such a $w$ is equivalent to the intersection $\bigcap_{l\in L}B_t(l)$ being non-empty or in other words, the Dowker nerve of $\Lambda$ is the usual ambient Čech complex of $L$.
\end{ex}

\begin{ex}[Multicover filtration]
	Consider the poset $P=\N^{op}\times[0,\infty)$ and let $L=\mathcal{P}(X)$ where $X\subset W=\R^d$. Define the Dowker dissimilarity $\Lambda\colon L\times W\to P^*$ be the Dowker dissimilarity defined by
	$$
	\Lambda(A,y)(k,t) := \begin{cases}
		1\quad\text{if }(k,t)\geq(|A|,\,d_H(\{y\},A))\text{ and}\\
		0\quad\text{otherwise.}
	\end{cases}
	$$
	
	Since $d_H(\{y\}, A)=\max_{a\in A}d(y,a)$, we see that
	$$
	B_\Lambda(A,(k,t)):=\begin{cases}
		\bigcap_{a\in A}B_t(a)\quad\text{if }k\leq |A| \text{ and}\\
		\emptyset\quad\text{otherwise.}
	\end{cases}
	$$
	
	Following the notation in \autocite{buchet2023sparse}, let $\binom{X}{k}$ denote the set of all subsets $A\subseteq X$ with $|A|=k$, and define the lens $L_t(A)$ corresponding to $A\in\binom{X}{k}$ by $L_t(A)=\bigcap_{a\in A}B_t(a)$. The \textit{$k$-fold cover} $L(t,X)$ of $X$ at scale $t$ is defined as $L(t,X)=\bigcup_{A\in\binom{X}{k}}L_t(A)$. The \textit{$k$th order Čech complex with radius} $t$ over $X$, denoted by $\text{\v{C}ech}_t(X,k)$, is defined as the nerve of the covering $\{L_t(A)\}_{A\in\binom{X}{k}}$ of $L(t, X)$.
	
	The Dowker nerve $N\Lambda_{(k,t)}$ is homotopy equivalent to the higher order Čech complex $\text{\v{C}ech}_t(X,k)$. By the nerve lemma, the Dowker nerve is homotopy equivalent to the union 
	$$
	\bigcup_{A\in\mathcal{P}(X)}B_\Lambda(A,(k,t)) = \bigcup_{\substack{A\in\mathcal{P}(X)\\|A|\geq k}}\bigcap_{a\in A}B_t(a) = \bigcup_{A\in\binom{X}{k}}L_t(A) = L(t,X)
	$$
	and $L(t,X)\simeq\text{\v{C}ech}_t(X,k)$, again by application of the nerve lemma. (The second equality above follows from the fact that if $|A|>k$ then we can remove points from $A$ to obtain $A'$ with $|A'|=k$ and $B_\Lambda(A,(k,t))\subset B_\Lambda(A',(k,t))$.)
	
	Note that $\text{\v{C}ech}_t(X,k)\subsetneq N\Lambda_{(k,t)}$ since the two complexes have different vertex sets $\binom{X}{k}$ and $\mathcal{P}(X)$, respectively.
\end{ex}

\section{Truncated nerves}

\begin{defn}[Translation function]
	A \textit{translation function} $\alpha\colon P\to P$ is an order preserving function which satisfies $p\leq\alpha(p)$ for all $p\in P$. 
\end{defn}

\begin{ex}
	Let $P=[0,\infty)$ and $a,b\in\R$ with $a>1$ and $b\geq 0$. Then $\alpha\colon t\mapsto at+b$ is a translation function.
\end{ex}

\begin{defn}[Interleaving]
	Given a translation function $\alpha\colon P\to P$ and a morphism $G\colon C\to C'$ of filtered objects in some category $\mathscr{C}$, then $G$ is said to be an \textit{$\alpha$-interleaving} if for every $p\in P$ there exists a morphism $F_p\colon C'_p\to C_{\alpha(p)}$ such that both triangles in the following diagram commute:
	\begin{center}
		\begin{tikzcd}
			C_p \arrow[r, "G_p"] \arrow[rd] & C'_p \arrow[d, "F_p"] \arrow[rd] & \\
			& C_{\alpha(p)} \arrow[r, "G_{\alpha(p)}"] & C'_{\alpha(p)}
		\end{tikzcd}
	\end{center}
	
\end{defn}

\begin{defn}[Truncation]
Given a dissimilarity $\Lambda\colon L\times W\to P^*$ and a translation function $\alpha\colon P\to P$, a function $T\colon L\to P$ is said to be an \textit{$\alpha$-truncation function for $\Lambda$} if for all $p\in P$ and all $l\in L$, there exists $l'\in L$ such that for all $w\in W$
\begin{equation}\label{eq:truncation-function-condition}
	w\in B_\Lambda(l,p) \implies w\in B_\Lambda(l',\alpha(p))\cap B_\Lambda(l',T(l')).
\end{equation}

The \textit{$T$-truncation} $\Gamma\colon L\times W\to P^*$ of a dissimilarity $\Lambda$ is the dissimilarity defined by letting
$$
\Gamma(l,w):=\begin{cases}
	\Lambda(l,w)\quad\text{if }w\in B_\Lambda(l, T(l)),\\
	\infty\quad\text{otherwise.}
\end{cases}
$$
\end{defn}

\begin{prop}
	Given  a dissimilarity $\Lambda\colon L\times W\to P^*$, a translation function $\alpha\colon P\to P$, and an $\alpha$-truncation function $T\colon L\to P$, let $\Gamma$ be the $T$-truncation of $\Lambda$. The inclusion of $N\Gamma$ in $N\Lambda$ is an $\alpha$-interleaving in the homotopy category of topological spaces.
\end{prop}

\begin{proof}
	Define the function $f_p\colon L\to L$ by picking for each $l\in L$ an element $l'\in L$ satisfying \cref{eq:truncation-function-condition} and set $f_p(l)=l'$. This induces a simplicial map $f_p\colon N\Lambda_p\to N\Gamma_{\alpha(p)}$. To see this, let $\sigma\in N\Lambda_p$ and pick some $w\in W$ such that $w\in B_\Lambda(l,p)$ for all $l\in\sigma$. By construction of $f_p$ we have $w\in B_\Lambda(l',\alpha(p))$ and $w\in B_\Lambda(l',T(l'))$ and hence $w\in B_\Gamma(l',\alpha(p))$. In other words, $f_p(\sigma)\in N\Gamma_{\alpha(p)}$ by the definition of the $T$-truncation. What is left to show, is that the triangles in the following diagram commute up to homotopy:
	\begin{center}
		\begin{tikzcd}
			N\Gamma_p \arrow[r, hook] \arrow[rd] & N\Lambda_p \arrow[d, "f_p"] \arrow[rd] & \\
			& N\Gamma_{\alpha(p)} \arrow[r, hook]    & N\Lambda_{\alpha(p)}
		\end{tikzcd}
	\end{center}
	Recall that two simplicial maps $f,g\colon K\to K'$ are homotopic if for all simplices $\sigma\in K$, the union $f(\sigma)\cup g(\sigma)$ is a simplex in $K'$, and that this induces a homotopy on the level of geometric realizations. It is therefore enough to show that
	\begin{enumerate}
		\item for all $\sigma\in N\Lambda_p$ we have $f_p(\sigma)\cup\sigma\in N\Lambda_{\alpha(p)}$, and
		\item for all $\sigma\in N\Gamma_p$ we have $f_p(\sigma)\cup\sigma\in N\Gamma_{\alpha(p)}$.
	\end{enumerate}
	1. This follows from the fact that for all $l'\in f_p(\sigma)$ we have $w\in B_\Lambda(l',\alpha(p))$ and the fact that for all $l\in\sigma$ we have $B_\Lambda(l,p)\subseteq B_\Lambda(l,\alpha(p))$ since $p\leq\alpha(p)$.
	
	2. Similar argument.
\end{proof}



\todo[inline, caption={}]{%
\begin{itemize}
	\item check definition 4.5 in \autocite{blaser2019sparse}.
	\item what does the \textit{farthest point sampling} look like in the multicover case?
\end{itemize}
}

\printbibliography
\end{document}